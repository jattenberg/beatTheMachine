\documentclass[letterpaper]{article}

\usepackage{aaai}
\usepackage[letterpaper,  top=0.75in, bottom=0.85in, left=0.75in, right=0.75in, twocolumn, pdftex]{geometry}

%\usepackage{times}
%\usepackage{helvet}
%\usepackage{courier}
\usepackage{url}
\usepackage{graphics}
\usepackage{graphicx}
\usepackage{amsmath}
\usepackage{amssymb}
\usepackage{subfigure}
%\usepackage{hyperref}
\frenchspacing

\newtheorem{example}{Example}
\newenvironment{xmpl}{\begin{example}\hspace{-.5em} \begin{textit}}{\end{textit}$\Box$\end{example}}


\newcommand{\todo}[1]{{\bf [TODO:{\em {#1}}]}}
\newcommand{\josh}[1]{{\bf [JOSH:{\em {#1}}]}}
\newcommand{\panos}[1]{{\bf [PANOS:{\em {#1}}]}}
\newcommand{\foster}[1]{{\bf [FOSTER:{\em {#1}}]}}
\newcommand{\drop}[1]{}

\pdfinfo{
/Title (Beat the Machine: Challenging workers to find the unknown unknowns)
/Author(Josh Attenberg, Panagiotis G. Ipeirotis, Foster Provost)
}

 
\title{Beat the Machine: Challenging Workers to Find the Unknown Unknowns}

\author{
Josh Attenberg \\
Polytechnic Institute of NYU \\
Brooklyn, NY \\
\texttt{josh@cis.poly.edu}
%\texttt{\href{mailto:josh@cis.poly.edu}{josh@cis.poly.edu}}
\And 
Panagiotis G.\ Ipeirotis \\
NYU Stern School of Business \\
New York, NY \\
\texttt{panos@stern.nyu.edu}
%\texttt{\href{mailto:panos@stern.nyu.edu}{panos@stern.nyu.edu}}
\And 
Foster Provost \\
NYU Stern School of Business \\
New York, NY \\
\texttt{fprovost@stern.nyu.edu}
%\texttt{\href{mailto:fprovost@stern.nyu.edu}{fprovost@stern.nyu.edu}}
}

\begin{document}

\maketitle


\begin{abstract}

  We present techniques for gathering data that expose errors of automatic predictive models.  In certain common settings, traditional methods for evaluating predictive models tend to miss rare-but-important errors---most importantly, rare cases for which the model is confident of its prediction (but wrong).  In this paper we present a system that, in a game-like setting, asks humans to identify cases that will cause the predictive-model-based system to fail. Such techniques are  valuable in discovering problematic cases that do not reveal themselves during the normal operation of the system, and may include cases that are rare but catastrophic. We describe the design of the system, including design iterations that did not quite work. In particular, the system incentivizes humans to provide examples that are difficult for the model to handle, by providing a reward proportional to the magnitude of the predictive model's error. The humans are asked to ``\emph{Beat the Machine}'' and find cases where the automatic model (``\emph{the Machine}'') is wrong. Experiments show that the humans using Beat the Machine identify more errors than traditional techniques for discovering errors in from predictive models, and indeed, they identify many more errors where the machine is confident it is correct.  Further, the cases the humans identify seem to be not simply outliers, but coherent areas missed completely by the model.  Beat the machine identifies  the ``unknown unknowns.''

\end{abstract}

\section{Introduction}
\label{sec:intro}


Many businesses and government organizations make decisions based on
estimations made by explicit or implicit models of the world.  Being
based on models, the decisions are not perfect.  Understanding the
imperfections of the models is important (i)~in order to improve the
models (where possible), (ii)~in order to prepare to deal with the
decision-making errors, and (iii)~in some cases in order to properly
hedge the risks.  However, a crucial challenge is that, for 
complicated decision-making scenarios, we often do not know where 
models of the world are imperfect and/or how the models' imperfections
will impinge on decision making. \emph{ We don't know what we don't know.}

We see the results of such failures of omniscience in grand
catastrophes, from terrorist attacks to unexpected nuclear disasters,
in mid-range failures, like cybersecurity breaches, and in failures of
operational models, such as predictive models for credit scoring,
fraud detection, document classification, etc.

Unknown unknowns are related to the classic contrast between reasoning
systems that make open- and closed-world
assumptions~\cite{Reiter77closedworld}: are the only answers to a query Q those
that are actually in the database?  In the context of predictive
modeling, in applications with limited labeled training data, small
disjuncts\cite{weiss10disjunct}, and possibly unknown selection biases, are
we willing to make the assumption that regularities that have no or
insufficient representation in the training data essentially do not
exist?

% In the context of data mining, one can
% think of learning methods as implicitly or explicitly making a
% closed-world assumption: 
% Whether
% learning systems truly make a closed-world assumption may be debatable
% under strict assumptions of learning theory and decision theory, where
% training data are sampled from the same population to which the model
% will be applied, and we know the costs of making errors.  However, in
% the real world we often learn and apply models anyway in the face of
% various sampling biases, often that we do not even know about.  Furthermore, 
% as we will describe, we often have 

In this paper we introduce and analyze a crowdsourcing system designed
to help uncover the ``unknown unknowns'' for predictive models.  The
system is designed to apply to settings where assessing the
performance of predictive models is particularly challenging.  Later we
will describe in detail the critical aspects of such settings, but
first let us introduce a motivating example to make the discussion
concrete.

Consider the following task: a firm has built a system for identifying
web pages that contain instances of ``hate speech'' (e.g., racist
content, antisemitism, and so on), based on a model that takes web
pages as input and produces as output a ``hate score.''  The firm
would like to use this system to help protect advertisers, who
(despite the best efforts of their advertising agents) sometimes see
their ads appearing adjacent to such objectionable content.  The
advertisers do not want their brands to be associated with such
content, and they definitely do not want to support such content,
explicitly or implicitly, with their ad dollars.  

How does this firm assess the strengths and weaknesses of its system
and model?  This scenario comprises a constellation of factors that
are not uncommon in organizational decision making, but are quite
problematic for conducting the assessment---particularly because of
the problem of unknown unknowns.  Specifically, this paper considers
applications where:

\begin{itemize}
\itemsep=0.0in
\item Every decision-making case can be represented by a description
  and a target.  We have a (predictive) model that can give us an estimate or
  score for the target for any case.  For this paper, we assume for
  simplicity that the target is binary, and that the truth would not
  be in dispute if known.\footnote{For our example, the
  description of the case would be the web page (its words, links,
  images, metadata, etc.).  The target would be whether or not it
  contains hate speech.}

\item We want to understand the inaccuracies of the
  model---specifically, the errors that it makes, and especially
  whether there are systematic patterns in the errors.  For example,
  is there a particular sort of hate speech that the model builders
  did not consider, and therefore the model misses it?

\item The process that is producing the data does not (necessarily)
  \textit{reveal} the target for free.  In our example, if we
  misclassify a hate speech page as being OK, we may never know.
  (Indeed, we usually never know.)  This is in contrast to
  \textit{self-revealing} processes; for example, in the case of credit-card
  fraud detection, we will eventually will be informed by the customer
  that there is fraud on her account.  For targeted marketing, we
  often eventually know whether the consumer responded to an offer or
  not.

\item Finally, there are important classes or subclasses of cases that
  are very rare, but nevertheless very important.  The rarity often is
  the very reason these cases were overlooked in the design of the
  system.  In our example, hate speech on the web itself is quite
  rare (thankfully).  Within hate speech, different subclasses are
  more or less rare.  Expressions of racial hatred are more common
  than expressions of hatred toward dwarves or data miners (both real cases).

\end{itemize}

These problem characteristics combine to make it extremely difficult to
discover system/model imperfections.  Just running the system, in
vitro or in vivo, does not uncover problems; as we do not observe
the true value of the target, we cannot compare the target to the model's
estimation or to the system's decision.

We \textit{can} invest in acquiring data to help us uncover
inaccuracies.  For example, we can task humans to score random or
selected subsets of cases.  Unfortunately, this has two major
drawbacks.  First, due to the rarity of the class of interest (e.g.,
hate speech) it can be very costly to find very few positive examples,
especially via random sampling of pages.  For example, hate speech
represents far less that $0.1\%$ of the population of web pages, with unusual or distinct forms of hate speech being far rarer still. Thus we would
have to invest in labeling more than 1000 web pages just to get one hate speech
example, and as has been pointed out
recently, often you need more than one label per page to get
high-quality labeling~\cite{shengKDD2008,raykar2009supervised}.


% whatever this means for model
% performance~\cite{forman2006quantifying}


In practice, we often turn to particular heuristics to identify
cases that can help to find the errors of our model.  There has been a
large amount of work studying ``active learning'' which attempts to
find particularly informative examples~\cite{SettlesActiveLearning}.
A large number of these strategies (uncertainty sampling, sampling
near the separating hyperplane, query-by-committee, query-by-bagging,
and others) essentially do the same thing: they choose the cases where
the model is least certain, and invest in human labels for these.
This strategy makes sense, as this is where we would think to find
errors.  Additionally, there has been a long history of understanding that
``near misses'' are the cases to use to best improve a model, both for
machine learning~\cite{winston1970learning} and for human
learning~\cite{vanlehn1998analogy}.

Unfortunately, although helpful in understanding and improving
modeling, for finding unknown unknowns, 
these strategies look exactly where we don't want to look.
These strategies explicitly deal with the ``known unknowns.''  The
model is uncertain about these examples---we ``know'' that we don't
know the answer for them (i.e., we have low confidence in the model's
output).  These strategies explicitly eschew, or in some cases
probabilistically downweight, the cases that we are
certain about, thereby \textit{reducing} the chance that we are going
to find the unknown unknowns.

With that substantial preamble, we can now state succinctly the goal
and contributions of this paper.  First, we describe the problem more
formally, including relationships to prior work.  We next discuss
changes to how we need to view the evaluation of classifiers, if we
want to move from a closed-world view of a predictive modeling problem
to an open-world view.  Then we introduce a technique and system to use
human workers to help find the \emph{unknown unknowns}.  Our
BeatTheMachine (BTM) system combines a game-like setup with incentives
designed to elicit cases where the model is confident and wrong.
Specifically, BTM rewards workers that discover cases that cause the
system to fail. The reward increases with the magnitude of the
failure. This setting makes the system to behave like a game,
encouraging steady, accurate participation in the tasks. We describe
our first experiences by the live deployment of this system, in a
setting for identifying web pages with offensive content on the
Internet. We show that this BTM setting discovers cases that are
inherently different than the errors identified by a random sampling
process. In fact, the two types of errors are very different. The BTM
process identifies ``big misses'' and potential catastrophic failures,
while traditional model-based example selection identifies ``near
misses'' that are more appropriate for fine-tuning the system.  The
evidence shows that BTM does not just find individual ``oddball''
outlier cases, but it finds systematic big errors.  In a sense, the
BTM process indeed gives us the opportunity to learn our ``unknown
unknowns'' and warn us about the failures that our current automatic
model cannot (yet) identify by itself.




\input{2-btm-design.tex}

\section{Experimental Studies}

To provide a first experimental evaluation of BTM, we asked two questions:
\begin{itemize}

\item Does BTM identify errors efficiently?

\item Can we use the discovered errors to improve the models?

\end{itemize}

For our experiments, we used the BTM system to challenge two
classification systems. One for detecting pages with hate speech, and
one for detecting pages with adult content. We ran the systems with
the configuration details described in the previous section (1 cent
for the base task, 50 cents maximum payment for a URL that generates
an error).

\textbf{Comparison with stratified random testing:} For the two systems, we compared BTM with the usual quality assurance process of examining the output of the classifier to identify errors.  Examining a uniform random sample of the output is particularly uninformative, as the classifiers are quite accurate and the distributions are quite unbalanced, and so the vast majority of cases are correctly classified and not objectionable.  Therefore, standard procedure is to examine a random sample, stratified by the model's confidence score.  Specifically, the range of confidence scores [0,1] was divided into $k$ equal-width bins.  A set of $N$ URLs for testing was sampled randomly, with $\frac{N}{k}$ from each bin.  This stratification is used because it generally finds more errors, because it over-samples the URLs for which the models have low confidence (and are likely to be wrong).  However, the discovered errors are likely to be ``known unknowns.''  

For the adult classifier, the human workers identified errors in 16\% of the inspected cases (\textit{much} higher than the natural error rate of the classifier).  In contrast, using BTM, more than 25\% of the submitted cases generated an error (a 56\% increase). The corresponding statistics for hate speech were even better: workers identified errors only in 9\% of the inspections for stratified random sampling, but they identified errors in 27\% of the URLs with BTM. These results indicate that the BTM process is indeed more efficient than the standard evaluation procedure in identifying problematic cases.  It should be noted that we could increase the ``efficiency'' of the non-BTM procedure by simply sampling more from the low-confidence cases.  However, this would directly reduce the number of ``unknown unknowns'' discovered.  At the extreme, the largest number of errors would be found by sampling only in the low-confidence region.  All the errors found would then be known unknowns.  So, let's now consider the effect of BTM on the severity of the errors found.

\textbf{Comparing the severity of errors:} Figure~\ref{fig:hate-speech} and~\ref{fig:adult} show the distribution of errors for hate speech and adult content, respectively. A consistent behavior is observed for both categories: BTM identifies a significantly larger number of severe misses---the unknown unknowns. Within the errors identified by BTM, 25\% were cases of high severity; the model was confident that it was making the correct decision (classifying the content as benign, with 100\% confidence), but in reality the decision was incorrect. So, not only does BTM identify a larger number problematic cases than the stratified testing, but also a significant number of these cases were unknown unknowns: cases that would be missed and without a very unpleasant event (possibly a catastrophe), we never would know that we missed them. In contrast, and by now as expected, most of the identified 
errors for the stratified random sampling were near misses that occur near the decision boundary.

\begin{figure}[t]
\centering
\center{
\subfigure[Hate Speech]{
\includegraphics[width= \columnwidth]{plots/Hate-speech-scores.PNG}
\label{fig:hate-speech}
}
\subfigure[Adult Content]{
\includegraphics[width= \columnwidth]{plots/porn-scores.PNG}
\label{fig:adult}
}
\caption{Distributions of the magnitude of the identified errors by BTM and by random sampling for two ad safety tasks}
}
\label{fig:results}
\end{figure}

\drop{
\begin{figure}[t]
\center{\includegraphics[width=\columnwidth]{plots/Hate-speech-scores.PNG}}
\caption{A distribution of the magnitude of the identified errors by BTM and by random sampling for the hate speech category.}
\label{fig:hate-speech}
\end{figure}

\begin{figure}[t]
\center{\includegraphics[width=\columnwidth]{plots/porn-scores.PNG}}
\caption{A distribution of the magnitude of the identified errors by BTM and by random sampling for the adult content category.}
\label{fig:adult}
\end{figure}
}

\textbf{Learning from identified errors:} The next, natural question is whether the identified erroneous decisions could be used to improve the decision models.  This actually is a very complicated problem, and a thorough treatment is beyond the scope of this short paper. For example, oversampling cases where a model makes big mistakes can be catastrophic for 
learning (think simply about oversampling outliers in a linear regression).
On the other hand, techniques like boosting \cite{Freund99ashort} have gotten tremendous
advantage by overweighting cases where the current model is incorrect.

Nevertheless, we can offer some initial insights. We can examine whether the cases found by BTM seem to be isolated outliers, or whether they seem to be regularities that can be modeled. To this end we ran the following experiment: We attempted to learn a model that would classify positive and negative examples from amongst the BTM-identified cases.\footnote{That is, false negatives and false positives from model being considered, respectively } Internal consistency in the identified errors would suggest that these cases are not outliers, but rather constitute parts of the space where the model fails systematically (potentially without being aware of the failures).

Figure~\ref{fig:curves} shows the results of this process. The ``btm only'' line shows the quality of the model built and tested using the error cases identified by the BTM process.  The ``student only'' line shows the quality of the model built and tested using examples gathered through stratified random sampling (the pages selected through random sampling were inspected by students, hence the name). Both the btm-only and student-only lines show quality measurements computed via cross-validation.  The results show that the quality of the models is fairly high, illustrating that there is consistency and internal coherence in these sets pages.  The fact that the BTM model can reach high levels of accuracy indicates that BTM indeed identifies systematic errors, and not just disparate outliers.  The comparatively lower quality of the random sampling model also illustrates that these pages are inherently more difficult to learn from; this is consistent with our discussion above that the discovery via stratified random sampling (DVSRS) focuses on the ambiguous cases (those that the current model is uncertain about), while BTM discovers incorrectly classified areas of the space that have been systematically ignored.

We also can examine whether the two approaches (DVSRS and BTM) identify sets of similar examples, or whether each of them identifies something completely different. For that, we tested the performance of BTM using the examples from DVSRS (``student'') and vice versa. The results indicate that there is little cross-consistency between the models. What we discover using BTM has little effectiveness on the error cases identified through DVSRS, and vice versa. This finding indicates  that BTM reveals errors in parts of the space unexplored by DVSRS.

BTM and DVSRS seem to be different processes, capable of identifying different types of errors. Each of these has its place in the evaluation and improvement of automatic models. DVSRS identifies cases where the model already knows that it is not confident. The BTM process, through its game-like structure and probing nature, encourages the discovery of unknown problems in the model. The fact that humans can easily find challenging cases for the automatic models, when being themselves confronted with this challenge, also indicates that human expertise and curiosity can improve even very accurate automatic models.



\begin{figure}[t]
\center{\includegraphics[width=\columnwidth]{plots/adt_btm_eval.png}}
\caption{Learning curves generated by the models
using cross-validation (BTM and student lines), and then use as test case for BTM the errors identified by random sampling (BTM on students), and vice versa (students on BTM).}
\label{fig:curves}
\end{figure}




\section{Current and Future Research}

We discussed and explored the design of the \emph{Beat the Machine} process for directly integrating humans into testing automatic decision models for vulnerabilities. Our results suggest that BTM is especially good in identifying cases where the model fails, while being confident that it is correct.  It is naturally interesting to examine how to best use knowledge of such vulnerabilities to improve the automatic decisions models. 


Vulnerability testing is common in areas of computer security, where ``white hat'' hackers with the appropriate expertise try to expose vulnerabilities in the security infrastructure of a firm. In our setting, we see that even lay users can easily find unknown holes in automatic decision models that test very well in ``standard'' tests, and show high classification performance when measured with the traditional, usual metrics (accuracy, AUC, etc).  Thus, builders of automatic decision models should take extra care when using these
traditional metrics for evaluations.

In our live deployment, untrained humans, with the appropriate incentives, were able to ``beat the machine'' seemingly easily, and discover a large number of vulnerabilities. This is, of course, useful by itself: the ``unknown unknowns'' become ``known unknowns'' and we can prepare to deal with these cases. But the key question for future research is also: how can we best incorporate such knowledge so that both ``unknown unknowns'' and ``known unknowns'' become ``known knowns.''

\section*{Acknowledgements}

The authors thank George A. Kellner and NEC for faculty fellowships,
and AdSafe Media for expertise, support, and data.  The models used in
this paper are not necessarily models used in production by any
company. This work was partially supported by the National Science 
Foundation under Grant No. IIS–0643846.


% [Add ambiguity in target variable to Limitations]

% ***To add to current/future work below:

%- How can we actually improve models with these unknown unknown cases?
%  Just plunking them into a training set yields mixed results.  This
%  may be because the training gets skewed, and we need to have a
%  specially designed training system.  Or it may be because we do not
%  really know how to evaluate the ``improved'' system.  What should be
%  the composition of the test set exactly?

%- In KDD-2010 we introduced guided learning, and showed that it can be
%  very useful for quickly building models in domains such as this.  We
%  are in the process of performing a similar comparison of BTM to GL.
%  Notably, GL also is not focused at all on the hard-to-envision
%  cases; in contrast, the incentive system there is to give easy to
%  find cases.

% For discussion:

% - relate to scenario planning
% - relate to white-hat hackers


\small{
\bibliographystyle{aaai}
\bibliography{activeinference} 
}


\end{document}