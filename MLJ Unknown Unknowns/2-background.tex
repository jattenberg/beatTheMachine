\section{Background and Scope}

Many businesses and government organizations make decisions based on
estimations made by explicit or implicit models of the world.  Being
based on models, the decisions are not perfect.  Understanding the
imperfections of the models is important (i)~in order to improve the
models (where possible), (ii)~in order to prepare to deal with the
decision-making errors, and (iii)~in some cases in order to properly
hedge the risks.  However, a crucial challenge is that, for 
complicated decision-making scenarios, we often do not know where 
models of the world are imperfect and/or how the models' imperfections
will impinge on decision making. \emph{ We don't know what we don't know.}

We see the results of such failures of omniscience in grand
catastrophes, from terrorist attacks to unexpected nuclear disasters,
in mid-range failures, like cybersecurity breaches, and in failures of
operational models, such as predictive models for credit scoring,
fraud detection, document classification, etc.

% In this paper we introduce and analyze a crowdsourcing system designed
% to help uncover the ``unknown unknowns'' for predictive models.  The
% system is designed to apply to settings where assessing the
% performance of predictive models is particularly challenging.  Later we
% will describe in detail the critical aspects of such settings, but
% first let us introduce a motivating example to make the discussion
% concrete.

Specifically, this paper considers applications where:

\begin{itemize}
\itemsep=0.0in
\item Every decision-making case can be represented by a description
  and a target.  We have a (predictive) model that can give us an estimate or
  score for the target for any case.  For this paper, we assume, for
  simplicity and without loss of generaliry, that the target is binary, and that the truth would not
  be in dispute if known.\footnote{For our example, the
  description of the case would be the web page (its words, links,
  images, metadata, etc.).  The target would be whether or not it
  contains hate speech.}

\item We want to understand the inaccuracies of the
  model---specifically, the errors that it makes, and especially
  whether there are systematic patterns in the errors in regions of the 
  space where the model is confident about its decisions and provides a very 
  low estimate for misclassification costs.  For example,
  is there a particular sort of hate speech that the model builders
  did not consider, and therefore the model misses it, while at the same
  time being confident about the reported decision?

% PANOS: Is the self-revealing property important?

% \item The process that is producing the data does not (necessarily) \textit{reveal} the target for free.  In our example, if we misclassify a hate speech page as being OK, we may never know. (Indeed, we usually never know.)  This is in contrast to \textit{self-revealing} processes; for example, in the case of credit-card fraud detection, we will eventually will be informed by the customer that there is fraud on her account.  For targeted marketing, we often eventually know whether the consumer responded to an offer or not.

% PANOS: Is the class-imbalance property important?

\item Finally, there are important classes or subclasses of cases that
  are very rare, but nevertheless very important.  The rarity often is
  the very reason these cases were overlooked in the design of the
  system.  In our example, hate speech on the web itself is quite
  rare (thankfully).  Within hate speech, different subclasses are
  more or less rare.  Expressions of racial hatred are more common
  than expressions of hatred toward dwarves or data miners (both real cases).

\end{itemize}

These problem characteristics combine to make it extremely difficult to
discover system/model imperfections.  Just running the system, in
vitro or in vivo, does not uncover problems; as we do not observe
the true value of the target, we cannot compare the target to the model's
estimation or to the system's decision.

We \textit{can} invest in acquiring data to help us uncover inaccuracies.  For example, we can task humans to score random or selected subsets of cases.  Unfortunately, this has two major drawbacks.  First, due to the rarity of the class of interest (e.g., hate speech) it can be very costly to find very few positive examples, especially via random sampling of pages.  For example, hate speech represents far less that $0.0001\%$ of the population of (ad supported) web pages, with unusual or distinct forms of hate speech being far rarer still. Thus we would have to invest in labeling more than a million randomly selected web pages just to get one hate speech example, and as has been pointed out recently, often you need more than one label per page to get high-quality labeling~\cite{shengKDD2008,raykar2009supervised}.


% whatever this means for model
% performance~\cite{forman2006quantifying}


In practice, we often turn to particular heuristics to identify
cases that can help to find the errors of our model.  There has been a
large amount of work studying ``active learning'' which attempts to
find particularly informative examples~\cite{SettlesActiveLearning}.
A large number of these strategies (uncertainty sampling, sampling
near the separating hyperplane, query-by-committee, query-by-bagging,
and others) essentially do the same thing: they choose the cases where
the model is least certain, and invest in human labels for these.
This strategy makes sense, as this is where we would think to find
errors.  Additionally, there has been a long history of understanding that
``near misses'' are the cases to use to best improve a model, both for
machine learning~\cite{winston1970learning} and for human
learning~\cite{vanlehn1998analogy}.

Unfortunately, although helpful in understanding and improving
modeling, for finding unknown unknowns, 
these strategies look exactly where we don't want to look.
These strategies explicitly deal with the ``known unknowns.''  The
model is uncertain about these examples---we ``know'' that we don't
know the answer for them (i.e., we have low confidence in the model's
output).  These strategies explicitly eschew, or in some cases
probabilistically downweight, the cases that we are
certain about, thereby \textit{reducing} the chance that we are going
to find the unknown unknowns.

In what follows, we next discuss
changes to how we need to view the evaluation of classifiers, if we
want to move from a closed-world view of a predictive modeling problem
to an open-world view.  Then we introduce a technique and system to use
human workers to help find the \emph{unknown unknowns}.  Our
BeatTheMachine (BTM) system combines a game-like setup with incentives
designed to elicit cases where the model is confident and wrong.
Specifically, BTM rewards workers that discover cases that cause the
system to fail. The reward increases with the magnitude of the
failure. This setting makes the system to behave like a game,
encouraging steady, accurate participation in the tasks. We describe
our first experiences by the live deployment of this system, in a
setting for identifying web pages with offensive content on the
Internet. We show that this BTM setting discovers cases that are
inherently different than the errors identified by a random sampling
process. In fact, the two types of errors are very different. The BTM
process identifies ``big misses'' and potential catastrophic failures,
while traditional model-based example selection identifies ``near
misses'' that are more appropriate for fine-tuning the system.  The
evidence shows that BTM does not just find individual ``oddball''
outlier cases, but it finds systematic big errors.  In a sense, the
BTM process indeed gives us the opportunity to learn our ``unknown
unknowns'' and warn us about the failures that our current automatic
model cannot (yet) identify by itself.


